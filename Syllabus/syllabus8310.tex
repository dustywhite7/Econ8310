\documentclass[12pt, margin=.5in]{article}

\usepackage{fontspec}
\usepackage{hyperref}

\defaultfontfeatures{Mapping=tex-text}
%\setmainfont {Adobe Garamond Pro} % Main document font
%\setsansfont {Gill Sans} % Used in the from address line above the to address


\pagestyle{empty}

\begin{document}
\vspace*{-6em}
\begin{center}
{\Large ECON 8310 (BSAD 8080)\   \ -- \ Business Forecasting \\[.5em] Date \& Time:  Thursdays, 6-8:40 PM, Room:  MH 115   
}
\end{center}

\setlength{\unitlength}{1in}

\hspace*{-4em}\begin{picture}(6,.1) 
\put(0,0) {\line(1,0){6.25}}         
\end{picture}

 

\renewcommand{\arraystretch}{2}

\begin{itemize}
\vskip.25in
\item[\textbf{Instructor:}] Dustin White, PhD\\  MH 332M\\ Phone: 402-554-3303\\Email: \href{mailto:drwhite@unomaha.edu}{drwhite@unomaha.edu}\\Slack Channel: \href{https://datascienceuno.slack.com}{https://datascienceuno.slack.com}
\vskip.25in
\item[\textbf{Office Hours:}] Thursdays from 5-6 PM, and by appointment.

\vskip.25in
\item[\textbf{Materials:}] Course Slides (hosted on Github at \href{https://github.com/dustywhite7/Econ8310}{https://github.com/dustywhite7/Econ8310})\\\\ Python (I recommend Anaconda, since it comes prepackaged with most of the numeric and analytic libraries we will use)\\\\ Mimir License (\$25)

\vskip.25in
\item[\textbf{Prerequisites:}]
ECON 8320 (or equivalent programming experience) AND ECON 8300 (or equivalent multivariate regression analysis coursework)\\
OR Instructor Approval\\

We will be creating our own forecasts, and my examples will use Python. You are welcome to use any other programming language you prefer (such as R), but \textbf{I will not provide support for other software or languages}.

\vskip.25in
\item[\textbf{Description:}]
The course will cover forecasting tools and applications applied to business settings. We will cover traditional Econometric forecasting methods in the first half of the class. In the second half of the course, we will focus on models in predictive analytics and machine learning, since these models are quickly becoming critical tools for forecasters in many settings. The course will include lecture and lab time, and labs will be focused on teaching students how to implement the models discussed in lectures.


\vspace*{.15in}
\item[\textbf{Course Outline:}]

\textbf{Part 1: Time Series Models}\hfill \\[1.8em]
Review of OLS, Tools for Class \dotfill 1 day\\
Time Series Models - ARIMA \dotfill 1 day\\
Time Series Models - ARIMAX \dotfill 1 day\\
Time Series Models - VAR \dotfill 1 day\\
Time Series Models - GAM \dotfill 1 day\\
Panel Data Models - Fixed-Effects Model \dotfill 1 day\\
MIDTERM EXAM \dotfill 2 days\\[1.8em]
\textbf{Part 2: Predictive Models}\hfill \\[1.8em]
Logistic Regression \dotfill 1 day\\
Lasso and Feature Selection \dotfill 1 day\\
Decision Trees \dotfill 1 day\\
Ensemble Methods - Random Forests \dotfill 1 day\\
Support Vector Machines \dotfill 1 day\\
$k$-Nearest Neighbors \dotfill 1 day\\
FINAL EXAM \dotfill 2 days\\



\vspace*{.15in}
\item[\textbf{Grade Policy:}] 
Reading Summaries (12 total) \dotfill 120 points\\
Homework/Lab Assignments (4 total) \dotfill 360 points\\
Midterm Exam/Presentation\dotfill 250 points \\
Final Exam/Presentation \dotfill 270 points \\

\vspace*{1em}
Grades will be distributed according to the following grade scale: \\

\begin{tabular}{l|l|l|l}
Score & Letter Grade & Score & Letter Grade\\
\hline
A & > 939 & C+ & 775 - 799 \\
A- & 900 - 939 & C & 725 - 774 \\
B+ & 875 - 899 & C- & 700 - 724 \\
B & 825 - 874 & D & 600 - 699 \\
B- & 800 - 824 & F & < 600 \\
\end{tabular}


\vskip.25in
\item[\textbf{Course Objectives}:] After this course, students should be capable of:
\begin{enumerate}
\item Understanding the respective strengths and weaknesses of the models presented in class
\item Implementing Econometric forecasting models
\item Applying machine learning algorithms in a forecasting setting 
\end{enumerate}

\vskip.25in
\item[\textbf{Grading}:] All assignments are to be submitted through the appropriate dropboxes on the course website. Rubrics will be posted, and will contain detailed information on the assignment grading policy. 

\vskip.25in
\item[\textbf{Homework}:]  Late work is not accepted, except as outlined in University policy.

\vskip.25in
\item[\textbf{Academic Honesty}:]  UNO’s requirements for Academic Integrity and Behavior All students are required to adhere to the highest standards of academic integrity and behavior and must satisfy the  \textbf{\href{http://www.unomaha.edu/student-life/student-conduct-and-community-standards/policies/academic-integrity.php}{UNO Academic Integrity Policy}} and  \textbf{\href{http://www.unomaha.edu/student-life/student-conduct-and-community-standards/policies/code-of-conduct.php}{Student Code of Conduct}}. It is the student’s responsibility to read, understand and abide by these policies. If I find that you have plagiarized, been dishonest in completing your assignments, or cheated an an exam or assignment, then I reserve the right to award you no points on the entire exam, project, or assignment and to report the behavior to the university. If this behavior is repeated, I reserve the right to award a failing grade, independent of your score on other assignments. Academic integrity is essential to education, and I take it very seriously.

\vskip.25in
\item[\textbf{Extra Help}:]  Dot not hesitate to come to my office during office hours or by appointment
to discuss a homework problem or any aspect of the course. The longer you wait, the more you will struggle.



\end{itemize}







\end{document}
